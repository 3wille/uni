%Einfache Vorlage fŸr eine mit Latex realisierte Hausarbeit von http://www.studieren-info.de
%Du kannst diese Vorlage fŸr deine Hausarbeit beliebig anpassen%


%-------------------
%Beginn des Kopfbereiches
%-------------------

%Wir verwenden eine DIN-A4-Seite und die Schriftgrš§e 13.
\documentclass[a4paper,13pt]{scrartcl} 


\usepackage{ucs}
\usepackage[utf8x]{inputenc}


%Diese drei Pakete benštigen wir fŸr die Umlaute, Deutsche Silbentrennung etc.
%Apple-Nutzer sollten anstelle von \usepackage[latin1]{inputenc} das Paket \usepackage[applemac]{inputenc} verwenden
%\usepackage[latin1]{inputenc}
\usepackage[ngerman]{babel}
\usepackage[T1]{fontenc}

%Das Paket erzeugt ein anklickbares Verzeichnis in der PDF-Datei.
\usepackage{hyperref}

%Das Paket wird fŸr die anderthalb-zeiligen Zeilenabstand benštigt
\usepackage{setspace}

%EinrŸckung eines neuen Absatzes
\setlength{\parindent}{0em}

%Definition der RŠnder
\usepackage[paper=a4paper,left=30mm,right=30mm,top=30mm,bottom=30mm]{geometry} 

%Abstand der Fu§noten
\deffootnote{1em}{1em}{\textsuperscript{\thefootnotemark\ }}

%Regeln, bis zu welcher Tiefe (section,subsection,subsubsection) †berschriften angezeigt werden sollen (Anzeige der †berschriften im Verzeichnis / Anzeige der Nummerierung)
\setcounter{tocdepth}{3}
\setcounter{secnumdepth}{3}

%-------------------
%Ende des Kopfbereiches
%-------------------




%-------------------
%Hier beginnt der Text deiner Hausarbeit
%-------------------
\begin{document}


%Beginn der Titelseite
\begin{titlepage}
\begin{small}
\vfill {Universität Hamburg \\ Fachbereich Informatik \\ Proseminar Lokale Rechner- und Mobilnetze \\ Sommersemester 2014 \\ Dr. Klaus-Dieter Heidtmann }
\end{small}


\begin{center}
\begin{Large}
\vfill {\textsf{\textbf{
Architekturen und Standards für Rechnernetze und Mobilnetze: 
Wireless Local Area Network
}}}
\end{Large}
\end{center}


\begin{small}
\vfill Juschua Stock \\ David Kirchhausen Monteiro \\ Frederik Wille \\
\today
\end{small}

\end{titlepage}
%Ende der Titelseite


%Inhaltsverzeichnis (aktualisiert sich erst nach dem zweiten Setzen)
\tableofcontents
\thispagestyle{empty}

%Beginn einer neuen Seite
\clearpage

%Anderthalbzeiliger Zeilenabstand ab hier
\onehalfspacing

\pagestyle{plain}


\section{Einleitung}

Was ist WLAN
Evtl Begriffserklärung



\section{Functionalität}

\subsection{Infrastruktur}

shit

\subsection{Ad-Hoc}

noch mehr shit

\section{technische Aspekte}
\subsection{IEEE}
Das Institute of Electrical and Electronics Engineers\footnote{Der volle Name wurde inzwischen fast komplett aufgegeben und wird nur noch für offizielle Dokumente genutzt} ist ein im Jahre 1963 gegründeter Berufsverband von Elektro- und Informationstechnik Ingenieuren. Neben den weitverbreiteten Standards veröffentlicht das IEEE einige sehr angesehene Fachzeitschriften, hält Tagungen und vergibt weltweit angesehene Auszeichnungen. \\
Organisatorisch sind die 425.000 Mitglieder der IEEE in 38 Societies gegliedert, die sich jeweils mit einem bestimmten Teilgebiet beschäftigt und wiederum in über 300 lokale Teilorganisationen unterteilt sind. Als Beispiel gibt es die IEEE Communications Society, die sich unter anderem mit WLAN beschäftigt. Für Gebiete in denen sich die Bereiche mehrerer Societies überschneiden, werden sogenannte Councils gegründet. Für die Entwicklung von Standards sind Komitees zuständig, wie "IEEE 802 LAN/MAN Standards Committee".\\
Um Spenden annehmen und verwalten zu können, wurde 1973 die IEEE Foundation gegründet. Ein Teil dieser Spenden kommen den Auszeichnungsträgern der IEEE zu, während der Rest für Entwicklungs-, Forschungs- und humanitäre Zwecke verwendet wird. \\
Ungefähr 30\% aller Fachzeitschriften in der Elektro- und Informationstechnik wird von der IEEE veröffentlicht, obwohl eine recht strenge Copyright Politik verfolgt wird. Jeder Autor eines in einer dieser veröffentlichen Zeitschriften muss sein gesamtes Copyright an die Organisation abtreten und darf nur für sich und seine Mitarbeiter Kopien behalten. Will der Autor seinen Artikel trotzdem weiter verbreiten, muss er die IEEE für das Copyright bezahlen und die IEEE entsprechend auf der ersten Seite vermerken.
\subsection{IEEE 802}
Die Standardisierungsfamilie 802 beinhaltet wohl die am weitesten verbreiteten Standards. In ihr sind alle Standards zusammengefasst, die sich mit dem Lokal Area Network und dem Metropolitan Area Network befassen. Sie umfasst 25 Standards, wobei zum Beispiel 802.15 das Personal Area Network in 6 Unterstandars geteilt ist. 
\section{Sicherheit}
asdf

\section{Schluss}
afgdfv

%Beginn einer neuen Seite
\clearpage

\section{Literaturverzeichnis}

Musterfrau, Renate: Muster. Frankfurt 2003.


Mustermann, Helmut: Noch ein Muster. Mit einer Einleitung hrsg. von Frank Muster. Frankfurt 2003.


\end{document}
%-------------------
%Hier endet der Text deiner Hausarbeit
%-------------------