%Einfache Vorlage fŸr eine mit Latex realisierte Hausarbeit von http://www.studieren-info.de
%Du kannst diese Vorlage fŸr deine Hausarbeit beliebig anpassen%


%-------------------
%Beginn des Kopfbereiches
%-------------------

%Wir verwenden eine DIN-A4-Seite und die Schriftgrš§e 13.
\documentclass[a4paper,13pt]{scrartcl} 


\usepackage{ucs}
\usepackage[utf8x]{inputenc}


%Diese drei Pakete benštigen wir fŸr die Umlaute, Deutsche Silbentrennung etc.
%Apple-Nutzer sollten anstelle von \usepackage[latin1]{inputenc} das Paket \usepackage[applemac]{inputenc} verwenden
%\usepackage[latin1]{inputenc}
\usepackage[ngerman]{babel}
\usepackage[T1]{fontenc}

%Das Paket erzeugt ein anklickbares Verzeichnis in der PDF-Datei.
\usepackage{hyperref}

%Das Paket wird fŸr die anderthalb-zeiligen Zeilenabstand benštigt
\usepackage{setspace}

%EinrŸckung eines neuen Absatzes
\setlength{\parindent}{0em}

%Definition der RŠnder
\usepackage[paper=a4paper,left=30mm,right=30mm,top=30mm,bottom=30mm]{geometry} 

%Abstand der Fu§noten
\deffootnote{1em}{1em}{\textsuperscript{\thefootnotemark\ }}

%Regeln, bis zu welcher Tiefe (section,subsection,subsubsection) †berschriften angezeigt werden sollen (Anzeige der †berschriften im Verzeichnis / Anzeige der Nummerierung)
\setcounter{tocdepth}{3}
\setcounter{secnumdepth}{3}

%-------------------
%Ende des Kopfbereiches
%-------------------




%-------------------
%Hier beginnt der Text deiner Hausarbeit
%-------------------
\begin{document}


%Beginn der Titelseite
\begin{titlepage}
\begin{small}
\vfill {Universität Hamburg \\ Fachbereich Informatik \\ Proseminar Lokale Rechner- und Mobilnetze \\ Sommersemester 2014 \\ Dr. Klaus-Dieter Heidtmann }
\end{small}


\begin{center}
\begin{Large}
\vfill {\textsf{\textbf{
Architekturen und Standards für Rechnernetze und Mobilnetze: 
Wireless Local Area Network
}}}
\end{Large}
\end{center}


\begin{small}
\vfill Juschua Stock \\ David Kirchhausen Monteiro \\ Frederik Wille \\
\today
\end{small}

\end{titlepage}
%Ende der Titelseite


%Inhaltsverzeichnis (aktualisiert sich erst nach dem zweiten Setzen)
\tableofcontents
\thispagestyle{empty}

%Beginn einer neuen Seite
\clearpage

%Anderthalbzeiliger Zeilenabstand ab hier
\onehalfspacing

\pagestyle{plain}


\section{Einleitung}

Was ist WLAN
Evtl Begriffserklärung



\section{Functionalität}

\subsection{Infrastruktur}

shit

\subsection{Ad-Hoc}

noch mehr shit

\section{technische Aspekte}
\subsection{IEEE}
Das Institute of Electrical and Electronics Engineers\footnote{Der volle Name wurde inzwischen fast komplett aufgegeben und wird nur noch für offizielle Dokumente genutzt} ist ein im Jahre 1963 gegründerter Berufsverband von Elektro- und Informationstechnik Ingenieuren. Neben den weitverbreiteten Standards veröffentlicht das IEEE einige sehr angesehene Fachzeitschriften und hält Tagungen. 
Untergliedert ist die IEEE in 
\subsubsection{UnterŸberschrift}
foobar
\subsubsection{UnterŸberschrift}
asfs
\section{Sicherheit}
asdf

\section{Schluss}
afgdfv

%Beginn einer neuen Seite
\clearpage

\section{Literaturverzeichnis}

Musterfrau, Renate: Muster. Frankfurt 2003.


Mustermann, Helmut: Noch ein Muster. Mit einer Einleitung hrsg. von Frank Muster. Frankfurt 2003.


\end{document}
%-------------------
%Hier endet der Text deiner Hausarbeit
%-------------------